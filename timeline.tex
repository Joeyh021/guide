\section{Timeline}

This module is comprised of approximately 30 lectures, 10 labs, 2 pieces of coursework, and an exam. This section contains a chronological, provisional schedule of all of these components. Each lecture aims to answer a specific question. You can test your understanding by asking yourself that question after each lecture and checking that you can answer it. 

There are typically three lectures per week. The definite timetable is available from the timetabling website\footnote{\url{https://timetablingmanagement.warwick.ac.uk/sws1819/}}, on the module website, or you can view your personal timetable on Tabula as well. 

\newcommand{\foo}{\makebox[0pt]{\textbullet}\hskip-0.5pt\vrule width 1pt\hspace{\labelsep}}

\newcommand{\LectureEntry}[4]{#1 & \begin{tabular}{p{11cm}}
		\textbf{#2} \\
		\emph{#3} \\
		#4
\end{tabular}}
\newcommand{\LabEntry}[3]{#1 & \begin{tabular}{p{11cm}}
		\textbf{#2} \\
		#3
\end{tabular}}

\begingroup
%\begin{table}
\newcommand{\oldarraystrech}{\arraystretch}
	\renewcommand\arraystretch{1.4}\vskip-1.5ex
	\begin{longtable}{@{\,}r <{\hskip 2pt} !{\foo} >{\raggedright\arraybackslash}p{12cm}}
		\addlinespace[1.5ex]
		\LabEntry{\practicalOneDate}{Lab 1: \practicalOneTitle}{\practicalOneAims} \\
		\LectureEntry{\lectureOneDate}{Lecture 1: \lectureOneTitle}{\lectureOneQuestion}{\lectureOneTopics} \\
		\LectureEntry{\lectureTwoDate}{Lecture 2: \lectureTwoTitle}{\lectureTwoQuestion}{\lectureTwoTopics} \\
		\LectureEntry{\lectureThreeDate}{Lecture 3: \lectureThreeTitle}{\lectureThreeQuestion}{\lectureThreeTopics} \\
		\LabEntry{\practicalTwoDate}{Lab 2: \practicalTwoTitle}{\practicalTwoAims} \\
		\LectureEntry{\lectureThreeBDate}{Lecture 4: \lectureThreeBTitle}{\lectureThreeBQuestion}{\lectureThreeBTopics} \\
		\LectureEntry{\lectureFourDate}{Lecture 5: \lectureFourTitle}{\lectureFourQuestion}{\lectureFourTopics} \\
		\LectureEntry{\lectureFiveDate}{Lecture 6: \lectureFiveTitle}{\lectureFiveQuestion}{\lectureFiveTopics} \\
		\LabEntry{\practicalThreeDate}{Lab 3: \practicalThreeTitle}{\practicalThreeAims} \\
		\LectureEntry{\lectureSixDate}{Lecture 7: \lectureSixTitle}{\lectureSixQuestion}{\lectureSixTopics} \\
		\LectureEntry{\lectureSevenDate}{Lecture 8: \lectureSevenTitle}{\lectureSevenQuestion}{\lectureSevenTopics} \\
		\LectureEntry{\lectureEightDate}{Lecture 9: \lectureEightTitle}{\lectureEightQuestion}{\lectureEightTopics} \\
		\LabEntry{\practicalFourDate}{Lab 4: \practicalFourTitle}{\practicalFourAims} \\
		\LectureEntry{\lectureNineDate}{Lecture 10: \lectureNineTitle}{\lectureNineQuestion}{\lectureNineTopics} \\
		\LectureEntry{\lectureNineBDate}{Lecture 11: \lectureNineBTitle}{\lectureNineBQuestion}{\lectureNineBTopics} \\
		\LectureEntry{\lectureTenDate}{Lecture 12: \lectureTenTitle}{\lectureTenQuestion}{\lectureTenTopics} \\
		\LabEntry{\practicalFiveDate}{Lab 5: \practicalFiveTitle}{\practicalFiveAims} \\
		\LectureEntry{\lectureElevenDate}{Lecture 13: \lectureElevenTitle}{\lectureElevenQuestion}{\lectureElevenTopics} \\
		\LectureEntry{\lectureElevenBDate}{Lecture 14: \lectureElevenBTitle}{\lectureElevenBQuestion}{\lectureElevenBTopics} \\
		\LectureEntry{\lectureTwelveDate}{Lecture 15: \lectureTwelveTitle}{\lectureTwelveQuestion}{\lectureTwelveTopics} \\
		\hline
		7 February & \begin{tabular}{p{13cm}}
			\textbf{Deadline: Coursework I} 
		\end{tabular}\\
		\hline
		\LabEntry{\practicalSixDate}{Lab 6: \practicalSixTitle}{\practicalSixAims} \\
		\LectureEntry{\lectureTwelveBDate}{Lecture 16: \lectureTwelveBTitle}{\lectureTwelveBQuestion}{\lectureTwelveBTopics} \\
		\LectureEntry{\lectureTwelveCDate}{Lecture 17: \lectureTwelveCTitle}{\lectureTwelveCQuestion}{\lectureTwelveCTopics} \\
		\LectureEntry{\lectureTwelveDDate}{Lecture 18: \lectureTwelveDTitle}{\lectureTwelveDQuestion}{\lectureTwelveDTopics} \\
		\LabEntry{\practicalSixBDate}{Lab 7: \practicalSixBTitle}{\practicalSixBAims} \\
		\LectureEntry{\lectureThirteenDate}{Lecture 19: \lectureThirteenTitle}{\lectureThirteenQuestion}{\lectureThirteenTopics} \\
		\LectureEntry{\lectureFourteenDate}{Lecture 20: \lectureFourteenTitle}{\lectureFourteenQuestion}{\lectureFourteenTopics} \\
		\LabEntry{\practicalSevenDate}{Lab 8: \practicalSevenTitle}{\practicalSevenAims} \\
		\LectureEntry{\lectureFifteenDate}{Lecture 21: \lectureFifteenTitle}{\lectureFifteenQuestion}{\lectureFifteenTopics} \\
		\LectureEntry{\lectureSixteenDate}{Lecture 22: \lectureSixteenTitle}{\lectureSixteenQuestion}{\lectureSixteenTopics} \\
		\LectureEntry{\lectureSeventeenDate}{Lecture 23: \lectureSeventeenTitle}{\lectureSeventeenQuestion}{\lectureSeventeenTopics} \\
		\LabEntry{\practicalEightDate}{Lab 9: \practicalEightTitle}{\practicalEightAims} \\
		\LectureEntry{\lectureEighteenDate}{Lecture 24: \lectureEighteenTitle}{\lectureEighteenQuestion}{\lectureEighteenTopics} \\
		\LectureEntry{\lectureEighteenBDate}{Lecture 25: \lectureEighteenBTitle}{\lectureEighteenBQuestion}{\lectureEighteenBTopics} \\
		\LectureEntry{\lectureNineteenDate}{Lecture 26: \lectureNineteenTitle}{\lectureNineteenQuestion}{\lectureNineteenTopics} \\
		\LabEntry{\practicalNineDate}{Lab 10: \practicalNineTitle}{\practicalNineAims} \\
		\LectureEntry{\lectureTwentyDate}{Lecture 27: \lectureTwentyTitle}{\lectureTwentyQuestion}{\lectureTwentyTopics} \\
		\LectureEntry{\lectureTwentyOneDate}{Lecture 28: \lectureTwentyOneTitle}{\lectureTwentyOneQuestion}{\lectureTwentyOneTopics} \\
		\LectureEntry{\lectureTwentyTwoDate}{Lecture 29: \lectureTwentyTwoTitle}{\lectureTwentyTwoQuestion}{\lectureTwentyTwoTopics} \\
		\hline
		14 March & \begin{tabular}{p{13cm}}
			\textbf{Deadline: Coursework II} 
		\end{tabular}\\
		\hline
		Term 3 & \begin{tabular}{p{13cm}}
			\textbf{Revision lectures}  \\
			Student-selected topics from the previous lectures.
		\end{tabular}\\
		Term 3 & \begin{tabular}{p{13cm}}
			\textbf{Exam}  \\
			2 hours. Answer any four out six questions.
		\end{tabular}
	\end{longtable}
%\end{table}
\endgroup