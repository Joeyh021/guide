\section{Timeline}
\label{sec:timeline}

This module is comprised of approximately 30 lectures, 10 labs, 2 pieces of coursework, and an exam. This section contains a chronological schedule of all of these components. Note that the schedule may be subject to changes due to \emph{e.g.} staff illness or other unforeseen circumstances. Each lecture aims to answer a specific question, which is shown in the timeline. You can test your understanding by asking yourself that question after each lecture and checking that you can answer it. 

There are typically three lectures per week. The definite timetable is available on Tabula\footnote{\url{https://tabula.warwick.ac.uk/profiles/department/cs/timetables?modules=cs141}}, on the module website, or you can view your personal timetable on Tabula as well. 

\newcommand{\foo}{\makebox[0pt]{\textbullet}\hskip-0.5pt\vrule width 1pt\hspace{\labelsep}}

\newcommand{\LectureEntry}[4]{#1 & \begin{tabular}{p{11cm}}
		\textbf{#2} \\
		\emph{#3} \\
		#4
\end{tabular}}
\newcommand{\LabEntry}[3]{#1 & \begin{tabular}{p{11cm}}
		\textbf{#2} \\
		#3
\end{tabular}}

\begingroup
%\begin{table}
\newcommand{\oldarraystrech}{\arraystretch}
	\renewcommand\arraystretch{1.4}\vskip-1.5ex
	\begin{longtable}{@{\,}r <{\hskip 2pt} !{\foo} >{\raggedright\arraybackslash}p{12cm}}
		\addlinespace[1.5ex]
		\LabEntry{6-10 January}{Week 1 exercises}{Relevant exercises for this week are \emph{Getting started}, \emph{Functions}, and \emph{Basic types}.} \\
		\LectureEntry{6 January}{Lecture 1: Introduction}{What is functional programming and why should we learn it?}{Overview of programming paradigms \& models of computation, examples of applications of functional programming, module overview, and recommended texts.} \\
		\LectureEntry{7 January}{Lecture 2: Definitions \& functions}{How do we write simple programs in Haskell?}{Definitions, basic arithmetic expressions, string values, boolean values, functions, using built-in functions, and basic pattern matching.} \\
		\LectureEntry{8 January}{Lecture 3: Basic types}{How does the compiler prevent us from writing bad software?}{Basic types, function types, parametric polymorphism, constraints, and pairs.} \\
		
		\LabEntry{13-17 January}{Week 2 exercises}{Relevant exercises for this week are \emph{Using the standard library}, \emph{Lists}, \emph{List comprehensions}, \emph{Recursive functions}, \emph{Higher-order functions}.} \\
		\LectureEntry{13 January}{Lecture 4: Lists}{How do we use lists in Haskell?}{Constructing lists, pattern-matching on lists, and list comprehensions.} \\
		\LectureEntry{14 January}{Lecture 5: Recursive functions}{How do we express loops without mutable state?}{Writing recursive functions for basic types (numbers, lists/strings), defining built-in functions ourselves.} \\
		\LectureEntry{15 January}{Lecture 6: Higher-order functions}{Can we write functions which abstract over common behaviours?}{Higher-order functions such as \haskellIn{map}, \haskellIn{filter}, etc., recursion primitives such as \haskellIn{foldr} and \haskellIn{foldl}.} \\
		
		\LabEntry{20-24 January}{Week 3 exercises}{Relevant exercises for this week are \emph{Data types} and \emph{Using other libraries}.} \\
		\LectureEntry{20 January}{Lecture 7: Data types \& type aliases}{How can we define our own types in Haskell?}{Type aliases, data types, data constructors, pattern matching on custom data constructors, recursion on values of custom types.} \\
		\LectureEntry{21 January}{Lecture 8: \emph{Fun with} functions}{What are some more examples of functions?}{Extended examples of using and defining functions.} \\
		\LectureEntry{22 January}{Lecture 9: Coursework I briefing}{What is the first coursework about?}{Demonstration of what the completed coursework will do, explanation of the rules, introduction to the skeleton code.} \\
		
		\LabEntry{27-31 January}{Week 4 exercises}{Relevant exercises for this week are \emph{Type classes}.} \\
		\LectureEntry{27 January}{Lecture 10: Type classes}{How can we restrict polymorphism and overload functions?}{Ad-hoc polymorphism via type classes, built-in type classes, and type class constraints.} \\
		\LectureEntry{28 January}{Lecture 11: \emph{Fun with} type classes}{How are type classes used in Haskell?}{Examples of type class and their instances.} \\
		\LectureEntry{29 January}{Lecture 12: Testing}{What tools are there for testing and how do we use them?}{Unit testing, property-based testing, and code coverage in Haskell.} \\
		
		\LabEntry{3-7 February}{Week 5 exercises}{Recommended exercises for this week are \emph{Proofs}.} \\
		\LectureEntry{3 February}{Lecture 13: Reasoning about programs}{Can we use formal reasoning techniques to prove properties about our programs?}{Equational reasoning, proofs by induction.} \\
		\LectureEntry{4 February}{Lecture 14: Reasoning about programs (cont.)}{Can we use formal reasoning techniques to prove properties about our programs?}{Equational reasoning, proofs by induction.} \\
		\LectureEntry{5 February}{Lecture 15: Constructive induction}{Can we use formal reasoning techniques to calculate more efficient programs?}{Using induction to calculate faster functions.} \\
		
		\hline
		6 February & \begin{tabular}{p{13cm}}
			\textbf{Deadline: Coursework I} 
		\end{tabular}\\
		\hline
		
		\LabEntry{10-14 February}{Week 6 exercises}{Relevant exercises for this week are \emph{Lazy evaluation}, \emph{Foldables}, and \emph{Functors}.} \\
		\LectureEntry{10 February}{Lecture 16: Lazy evaluation}{What order are programs evaluated in?}{Strict and lazy evaluation, calling conventions, benefits and disadvantages, and examples of lazy evaluation.} \\
		\LectureEntry{11 February}{Lecture 17: Foldables}{Are there any useful design patterns in functional programming?}{\haskellIn{Foldable} type class, its motivation, and examples.} \\
		\LectureEntry{12 February}{Lecture 18: Functors \& applicative functors}{Are there any other useful design patterns in functional programming?}{Functors, applicative functions, applications of applicative functors.} \\
		
		\LabEntry{17-21 February}{Week 7 exercises}{Relevant exercises for this week are \emph{Applicative functors}.} \\
		\LectureEntry{17 February}{Lecture 19: Functors \& applicative functors (cont.)}{Are there any other useful design patterns in functional programming?}{Functors, applicative functions, applications of applicative functors.} \\
		\LectureEntry{18 February}{Lecture 20: \emph{Fun with} applicative functors}{What can we do with applicative functors?}{Applicative functors in action.} \\
		\LectureEntry{19 February}{Lecture 21: Coursework II briefing}{What is the second coursework about?}{Demonstration of what the completed coursework will do, semantics of the programming language, introduction to the skeleton code.} \\
		
		\LabEntry{24-28 February}{Week 8 exercises}{Relevant exercises for this week are \emph{Effectful programming}.} \\
		\LectureEntry{24 February}{Lecture 22: Sequential composition}{How do structure programs in which one part of a program relies on the result of another part?}{Some functions for the sequential composition of \texttt{\small Maybe} values.} \\
		\LectureEntry{25 February}{Lecture 23: Sequential composition (cont.)}{Are there other examples of sequential composition?}{Some functions for the sequential composition of \texttt{\small State} values and some laws for sequential composition} \\
		\LectureEntry{26 February}{Lecture 24: \emph{Fun with} sequential composition}{How is sequential composition used in practice?}{Sequential composition in action.} \\
		
		\LabEntry{2-6 March}{Week 9 exercises}{Relevant exercises for this week are \emph{Input \& output} and \emph{Kinds}.} \\
		\LectureEntry{2 March}{Lecture 25: Input and output}{Can we write impure programs in a pure programming language?}{The \texttt{\small IO} monad.} \\
		\LectureEntry{3 March}{Lecture 26: \emph{Fun with} IO}{What do Haskell programs that make use of IO look like?}{The \texttt{\small IO} monad in action.} \\
		\LectureEntry{4 March}{Lecture 27: Type promotion \& GADTs}{How can we encode more information in types?}{Phantom types, GADTs, singleton types, pattern matching with GADTs.} \\
		
		\LabEntry{9-13 March}{Week 10 exercises}{Relevant exercises for this week are \emph{Type-level programming}.} \\
		\LectureEntry{9 March}{Lecture 28: Type families}{How can we perform computation at the type-level?}{Closed and open type families.} \\
		\LectureEntry{10 March}{Lecture 29: \emph{Fun with} type-level programming}{What are some examples of how type-level programming is used?}{Type-level programming in action.} \\
		\LectureEntry{11 March}{Lecture 30: Conclusions}{What have we learnt about functional programming?}{Summary of the module, information about the exam, and other general information.} \\
		\hline
		13 March & \begin{tabular}{p{13cm}}
			\textbf{Deadline: Coursework II} 
		\end{tabular}\\
		\hline
		Term 3 & \begin{tabular}{p{13cm}}
			\textbf{Revision lectures}  \\
			Student-selected topics from the previous lectures.
		\end{tabular}\\
		Term 3 & \begin{tabular}{p{13cm}}
			\textbf{Exam}  \\
			2 hours. Answer any four out of six questions.
		\end{tabular}
	\end{longtable}
%\end{table}
\endgroup
