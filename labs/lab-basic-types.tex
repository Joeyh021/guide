\section{Basic types}
\topics{Basic types, function types, parametric polymorphism, lists, pairs, and type class constraints.}

These exercises are about \emph{types} in Haskell. In a nutshell, the type of an expression tells you what sort of value an expression eventually evaluates to. Not all expressions can be typed! If an expression cannot be reduced to a normal form, such as \haskellIn{not 7}, then it does not have a type and the compiler will refuse to compile the program. We use the following notation to say that an expression \texttt{\small expression} has type \texttt{\small type} -- this is referred to as a \emph{typing}:
\begin{center}
	\begin{tabular}{lcl}
		\texttt{\small expression} & \texttt{\small ::} & \texttt{\small type}
	\end{tabular}
\end{center}
Some examples of typings are:
\begin{minted}{haskell}
True                       :: Bool
'a'                        :: Char 
\x -> x                    :: a -> a
\x -> False                :: a -> Bool 
\x -> \y -> y              :: a -> b -> b
if 5 > 6 then 'a' else 'b' :: Char 
42                         :: Num a => a
4 + 8 * 15 - 16            :: Num a => a
(\x -> x) True             :: Bool
\end{minted}
As you can see, the size or complexity of a term does not necessarily correspond to that of a type. It only matters what an expression evaluates to. Some expressions have multiple permissible types. For example, the expression \haskellIn{42} could have type \haskellIn{Int} (the type of signed integers where the precision depends on your platform), \haskellIn{Integer} (the type of arbitrary precision integers), \haskellIn{Num a => a} (a polymorphic type \haskellIn{a} constrained to all numeric types by the constraint \haskellIn{Num a}), as well as others.

\makebox[0.5cm]{\faBook}~\emph{Recommended reading}: Chapter 3 of \emph{Learn you a Haskell} \citep{lipovaca2011learn} or Chapter 3 of \emph{Programming in Haskell} \citep{hutton2016programming}.

\subsubsection{Types}

If an expression can be typed, the Haskell compiler can \emph{infer} the \emph{most general} type for us. For example, for numbers such as \haskellIn{42}, the \haskellIn{Int} type is permissible, but \haskellIn{Num a => a} is a more general type. ``More general'' generally means ``more polymorphic''. There is a command in the REPL which we can use to ask for the type of an expression:
\begin{center}
\begin{tabular}{|l|l|}
	\hline 
	\texttt{\small :t EXPRESSION}   & Shows the type of \texttt{\small EXPRESSION}. \\ 
	\hline 
\end{tabular}  
\end{center}

\taskLine

\task{Launch the REPL with \texttt{\small stack repl} in an arbitrary directory that does \emph{not} already contain any Haskell code and try it for yourself by asking for the types of the following expressions with the \texttt{\small :t} command:}
\begin{itemize}
	\item \haskellIn{'a'}
	\item \haskellIn{[True, True, False]}
	\item \haskellIn{[1,2,3,4,5]}
	\item \haskellIn{[]}
	\item \haskellIn{\x -> x}
	\item \haskellIn{1+1}
\end{itemize}

\task{Since the Haskell compiler always infers the most general type for an expression, you may sometimes want to validate whether a less general type is also permissible. You can annotate expressions with typings for that purpose and get the Haskell compiler to validate your annotations. For example, try to ask for the types of the following two expressions:}
\begin{itemize}
	\item \haskellIn{108}
	\item \haskellIn{(108 :: Int)}
	\item \haskellIn{(True :: Int)}
\end{itemize}
As you can see, you get the most general type for \haskellIn{108}. For \haskellIn{(108 :: Int)}, the compiler validates that \haskellIn{Int} is indeed a permissible type for \haskellIn{108}. For the last expression, we get a type error since \haskellIn{Int} is not a permissible type for \haskellIn{True}.

\taskLine

\task{Some expressions cannot be reduced to values and therefore do not have a type. Try asking for the types of the following expressions. Each expression will result in the type error which is explained below:} 
\begin{itemize}
\item \haskellIn{not 7} \\
As mentioned in the lecture on type classes, the literal \haskellIn{7} has the most general type \haskellIn{Num a => a}. The Haskell compiler also knows that the \haskellIn{not} function has type \haskellIn{Bool -> Bool} so it expects its argument to be of type \haskellIn{Bool}. Because \haskellIn{Bool} is less polymorphic than \haskellIn{a}, the Haskell compiler deduces that \haskellIn{7} should be of type \haskellIn{Bool}. However, this results in a constraint of \haskellIn{Num Bool} which cannot be resolved because there is no instance of \haskellIn{Num} for the \haskellIn{Bool} type.
\item \haskellIn{[1,True,3]} \\
This results in the same error.
\item \haskellIn{['a', False]} \\
Lists in Haskell are \emph{homogeneous}. That is, they can only contain elements which have the same type. Neither \haskellIn{'a'} nor \haskellIn{False} have a polymorphic type and both have different types. Therefore, the Haskell compiler will tell you that the two types do not match.
\end{itemize}

\taskLine

\task{The REPL essentially wraps each expression you try to evaluate into a call to the \haskellIn{show :: Show a => a -> String} function. Try to evaluate some expressions which \emph{are} well typed, but whose types do not satisfy the \haskellIn{Show} predicate. This is usually the case for functions. For example: }
\begin{itemize}
\item \haskellIn{not} \\
Even though \haskellIn{not} is well typed, you get a type error if you try to evaluate it in the REPL. That is because the REPL does not know how to display a result that is a function. The error you get will tell you that the \haskellIn{Show} constraint cannot be satisfied for \haskellIn{Bool -> Bool}, the type of \haskellIn{not}. 
\end{itemize}
