\section{Lab 1: \practicalOneTitle}
\topics{Definitions, basic arithmetic expressions, string values, boolean values, functions, using built-in functions, and basic pattern matching.}

%\newcommand{\task}[1]{\textbf{Task}: #1}

The goal of this first practical is twofold: firstly, you will learn to write some very simple Haskell programs and secondly, you will be able to familiarise yourself with the tools we use as part of this module, such as \bashIn{stack} to compile and test programs and \bashIn{git} for version control. The module guide contains instructions on how to use both programs.

If you have not done so yet, you should set up your Haskell development environment now (\Cref{sec:department-setup} and \Cref{sec:home-setup}). Most of the tools you will need are pre-installed on the departmental machines. If you are using one of the lab machines, you only need to run the following command in your shell (the Terminal application on the lab machines):
\begin{minted}{bash}
$ /modules/cs141/haskell-setup.sh
\end{minted}
There is some skeleton code for most of the lab sessions, including this one. You can obtain the code for this practical by cloning it from GitHub:
\begin{minted}{bash}
$ git clone https://github.com/fpclass/lab1
\end{minted}
By default, this will create a folder named \texttt{\small lab1} in the current working directory (your home folder, by default) with the skeleton code in it. Once you have cloned the repository, you may wish to verify that \bashIn{stack} compiles it without any problems:
\begin{minted}{bash}
$ cd lab1
$ stack build
\end{minted}
If everything goes well, you should see some output along the lines of:
\begin{minted}{text}
lab1-1.0.0.0: configure (lib)
Configuring lab1-1.0.0.0
lab1-1.0.0.0: build (lib)
Preprocessing library lab1-1.0.0.0
[1 of 1] Compiling Lab1      (src/Lab1.hs, ...)
lab1-1.0.0.0: copy/register
Installing library in ...
Registering lab1-1.0.0.0
\end{minted}
You are now ready to work on the exercises! A very useful tool for debugging which you may find helpful in completing the exercises is a Read-Eval-Print Loop (or REPL for short). This is offered by many modern programming languages and development tools, including the Glasgow Haskell Compiler (which \bashIn{stack} is using behind the scenes). You can launch the REPL by invoking the following command:
\begin{minted}{bash}
$ stack repl
\end{minted}
You can enter arbitrary expressions which the REPL will evaluate for you. You should see something similar to the following prompt:
\begin{minted}{haskell}
*Lab1 Lab1>
\end{minted}
Simply enter an expression like \haskellIn{1+1} and hit enter to evaluate it. The REPL will print the result of evaluating the expression:
\begin{minted}{haskell}
*Lab1 Lab1> 1+1
2
\end{minted}
There is a \texttt{\small src/Lab1.hs} file in the \texttt{\small lab1} directory which contains some definitions for this lab. Because we have run \bashIn{stack repl} in the directory with the code (\texttt{\small lab1}), the REPL has automatically loaded the the \texttt{\small src/Lab1.hs} file for us, so you can refer to definitions in it:
\begin{minted}{haskell}
*Lab1 Lab1> name
"Michael"
*Lab1 Lab1> age * 2
52
\end{minted}

\task{At this point you may wish to read \Cref{ch:tools} if you have not done so already and set up your text editor to suit your preferences. If you are using Atom, the \bashIn{haskell-setup.sh} script you ran earlier will already have installed some Haskell-related plugins for you.}

\task{Open the \texttt{\small src/Lab1.hs} file in your preferred text editor and modify the definitions of \haskellIn{age} and \haskellIn{name} to match your name and age.}

Instead of typing an expression which should be evaluated by the REPL, you may also type in a command (all commands start with a colon). The following commands are supported (among others):
\begin{center}
\begin{tabular}{|l|l|}
\hline 
    \texttt{\small :q}   & Quits the REPL. \\ 
\hline 
    \texttt{\small :r}  & Reloads all files that are currently loaded. \\ 
\hline 
    \texttt{\small :l FILENAME} & Loads \texttt{\small FILENAME} into the REPL. \\
\hline
\end{tabular} 
\end{center}
Assuming you did not close the REPL to edit \texttt{\small src/Lab1.hs}, it will still be running. The REPL does not automatically check for updates to any files that are currently loaded, so you will have to reload it with the \texttt{\small :r} command. In general, the \texttt{\small :r} command reloads all files that are loaded in the REPL. Now try evaluating \haskellIn{age} and \haskellIn{name} again. They should match whatever values you changed them to.

\task{Complete the definitions of \haskellIn{triple}, \haskellIn{tripleV2}, \haskellIn{not}, \haskellIn{and}, \haskellIn{max}, and \haskellIn{perimeterRect} in \texttt{\small src/Lab1.hs}. You can test them in the REPL to see if you have got them right. Remember to reload the file once you have made some changes.}

\task{Some of the above functions, such as \haskellIn{not}, \haskellIn{and}, and \haskellIn{max}, can be defined in many different ways. Aside from your current definitions for them, can you think of one additional way to define each?}

\task{In addition to the REPL, the skeleton code comes with a test suite which you can use to test your functions. Simply run \bashIn{stack test} in a terminal window to run all unit tests against your code. You should make sure that all tests succeed. You should see somewhere close to the end of the output:}
\begin{minted}{text}
7 examples, 0 failures
\end{minted}

\task{On paper (or equivalent), trace the evaluation of}
\begin{minted}{haskell}
min (perimeterRect 4 8) (perimeterRect 10 2)
\end{minted}

\task{With a friend, another student of your choice, or within a small group: each pick one topic from the lectures so far that you found confusing, then get one of the others to try and explain it after a few minutes of preparation.}

\task{With a friend, another student of your choice, or within a small group: discuss whether...}
\begin{itemize}
\item ...you think reduction or mutation is easier to understand?
\item ...you prefer to use indentation (Haskell, Python, ...) or curly brackets (Java, C, ...) to denote scope and why? 
\end{itemize}



