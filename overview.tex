%%%%%%%%%%%%%%%%%%%%%%%%%%%%%%%%%%%%%%%%%%%%%%%%%%%%%%%%%%%%%%%%%%%%%%%%%%%%%%%%
%% LaTeX sources for The Guide to Functional Programming
%% Michael B. Gale (m.gale@warwick.ac.uk)
%%
%% This work is licensed under the Creative Commons
%% Attribution-NonCommercial-ShareAlike 2.0 UK: England & Wales License. To
%% view a copy of this license, visit
%% http://creativecommons.org/licenses/by-nc-sa/2.0/uk/ or send a letter to
%% Creative Commons, PO Box 1866, Mountain View, CA 94042, USA.
%%%%%%%%%%%%%%%%%%%%%%%%%%%%%%%%%%%%%%%%%%%%%%%%%%%%%%%%%%%%%%%%%%%%%%%%%%%%%%%%

\chapter{Overview}

\emph{Functional Programming} is an optional module which follows on from modules such as CS118, CS132, or equivalents in other departments where you have learnt to write programs in the imperative style in languages such as C and Java. However, C and Java are just two of many programming languages and object-oriented programming is just one of many programming paradigms. You may think of programming languages as tools: a hammer is different from a screwdriver and both serve different purposes which they are good at. Programming languages are the same: different languages exist for different purposes and it is easier or harder to accomplish certain tasks in one or the other. To be a good programmer, you need to know which tools are at your disposal and when to use them.

In this module, you will learn about the functional programming paradigm. No prior programming knowledge is required and this module is suitable for most scientists. We will use Haskell, which is a lazy, purely functional programming language. Writing programs in Haskell is very different than writing programs in languages like Java and over the course of this module you will learn how to do that. In turn, this adds a powerful tool to your programming arsenal, you will gain a much deeper understanding of programming, and skills from this module can be applied in other languages, functional or not. In other words, you will become a better programmer!

This document serves as a companion to the module by giving you an overview of all the major components, including guidance on how to use the different tools you will encounter as part of this module. You can also find the coursework specifications as well as exercises for all of the labs in this guide.